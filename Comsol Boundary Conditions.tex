\documentclass[letterpaper,landscape, 10pt]{article}
\usepackage{mathtools}
\usepackage{geometry}

\geometry{margin=.75in}
\fontsize{10pt}{14pt}
\renewcommand\arraystretch{2.1}

\renewcommand{\vec}[1]{\underline{#1}}
\newcommand{\pd}[2]{\frac{\partial #1}{\partial #2}}
\newcommand{\transpose}[1]{#1^\text{T}}

\title{Trying to Understand the B.C.'s in Comsol}
\author{Pierce Hunter}
\date{2/1/2019}

\begin{document}
	\maketitle
	\section{Deriving $ 2\mu D $}
	\begin{gather*}
		D = \frac{1}{2}\left[\nabla\vec{v} + \transpose{\left(\nabla\vec{v}\right)}\right]\\
		\vec{v} = \left(V,W\right)\\
		\nabla\vec{v} = \begin{bmatrix}
		\displaystyle\pd{V}{x} & \displaystyle\pd{V}{y}\\
		\displaystyle\pd{W}{x} &\displaystyle \pd{W}{y}
		\end{bmatrix}\\
		\transpose{\left(\nabla\vec{v}\right)} = \begin{bmatrix}
		\displaystyle\pd{V}{x} &\displaystyle \pd{W}{x}\\
		\displaystyle\pd{V}{y} &\displaystyle \pd{W}{y}
		\end{bmatrix}\\
		\mu\left(\nabla\vec{v} + \transpose{\left(\nabla\vec{v}\right)}\right) = \begin{bmatrix}
		\displaystyle2\mu\pd{V}{x} & \displaystyle\mu\left(\pd{V}{y} + \pd{W}{x}\right)\\
		\displaystyle\mu\left(\pd{V}{y} + \pd{W}{x}\right) & \displaystyle2\mu\pd{W}{y}
		\end{bmatrix} = 2\mu D
	\end{gather*}
	\section{A Table of Normals and Boundary Conditions}
	\begin{center}
		\begin{tabular}{|c||c|c|c|c|}
			\cline{2-5}
			\multicolumn{1}{c||}{}& $ \vec{n} $ & $ U $ & $ V $ & $ W $\\
			\hline\hline
			Left & $ \left(-1,0\right)$ & $ U=0 $ & $ V=1-\left(1-Z\right)^4$ & $ W=0$ \\
			\hline
			Right & $ \left(1,0\right)$ & $ \displaystyle \mu\pd{U}{Y}=1$ & $ \displaystyle \pd{V}{Z} = 0$ & $ W=0$ \\
			\hline
			Top & $ \left(0,1\right)$ & $\displaystyle \mu\pd{U}{Z} = 0$ & $\displaystyle \mu\pd{V}{Z} = 0$ & \begin{tabular}{c}
				$ W = 0$\\
				$\displaystyle 2\mu\pd{W}{Z}-P+S' = 0$
			\end{tabular} \\
			\hline
			Ridge (No Slip) & $ \left(0,-1\right)$ & $ U=0$ & $ V=0$ & $ W=0$ \\
			\hline
			Ridge (Slip) & $ \left(0,-1\right)$ & $\displaystyle \begin{cases}
			\mu\pd{U}{Z} = \tau\frac{U}{\sqrt{U^2+\varepsilon^2V^2}}&\sqrt{U^2+\varepsilon^2V^2}>0\\
			\sqrt{\left(\mu\pd{U}{Z}\right)^2+\varepsilon^2\left(\mu\pd{V}{Z}\right)^2}\leq\tau & \sqrt{U^2+\varepsilon^2V^2}=0
			\end{cases} $ & $\displaystyle \begin{cases}
			\mu\pd{V}{Z} = \tau\frac{V}{\sqrt{U^2+\varepsilon^2V^2}}&\sqrt{U^2+\varepsilon^2V^2}>0\\
			\sqrt{\left(\mu\pd{U}{Z}\right)^2+\varepsilon^2\left(\mu\pd{V}{Z}\right)^2}\leq\tau & \sqrt{U^2+\varepsilon^2V^2}=0
			\end{cases} $ & $ W=0 $ \\
			\hline
			Stream & $ \left(0,-1\right)$ & $ \displaystyle \mu\pd{U}{Z} = 0 $ & $\displaystyle \mu\pd{V}{Z}=0 $ & $ W=0 $ \\
			\hline
		\end{tabular}
	\end{center}
	For conversion to comsol need $ Y \rightarrow x,\ Z \rightarrow y $
	\newpage
	\section{Downstream Velocity Boundary Conditions in Comsol}
	\subsection{Left Boundary}
	Dirichlet Boundary Condition:
	\begin{gather*}
		U = r\\
		r = 0 \implies U = 0
	\end{gather*}
	\subsection{Right Boundary}
	Flux/Source Boundary Condition:
	\begin{gather*}
		-\vec{n}\cdot\vec{\Gamma} = g-qu\\
		\vec{n} = \left(1,0\right)\\
		\vec{\Gamma} = \left(\mu\pd{U}{x},\mu\pd{U}{y}\right)\\
		-\vec{n}\cdot\vec{\Gamma} = -\mu\pd{U}{x}\\
		\text{Set: }g = -1,\ q = 0 \implies -\mu\pd{U}{x} = -1\\
		\mu\pd{U}{x} = 1
	\end{gather*}
	\subsection{Top Boundary}
	Zero Flux Boundary Condition:
	\begin{gather*}
		-\vec{n}\cdot\vec{\Gamma} = 0\\
		-\left(0,1\right)\cdot\left(\mu\pd{U}{x},\mu\pd{U}{y}\right) = 0\\
		-\mu\pd{U}{y} = 0\\
		\mu\pd{U}{y} = 0
	\end{gather*}
	\subsection{Bottom Boundary - Stream}
	Zero Flux Boundary Condition:
	\begin{gather*}
		-\vec{n}\cdot\vec{\Gamma} = 0\\
		-\left(0,-1\right)\cdot\left(\mu\pd{U}{x},\mu\pd{U}{y}\right) = 0\\
		\mu\pd{U}{y} = 0
	\end{gather*}
	\subsection{Bottom Boundary - Ridge (No Slip)}
	Dirichlet Boundary Condition:
	\begin{gather*}
		U = r\\
		r = 0 \implies U = 0
	\end{gather*}
	\subsection{Bottom Boundary - Ridge (Slip)}
	Flux/Source Boundary Condition:
	\begin{gather*}
		-\vec{n}\cdot\vec{\Gamma} = g-qu\\
		\vec{n} = \left(0,-1\right)\\
		\vec{\Gamma} = \left(\mu\pd{U}{x},\mu\pd{U}{y}\right)\\
		-\vec{n}\cdot\vec{\Gamma} = \mu\pd{U}{y}\\
		\text{Set: }g = \tau\frac{U}{\sqrt{U^2+\varepsilon^2V^2}},\ q = 0 \implies \mu\pd{U}{y} = \tau\frac{U}{\sqrt{U^2+\varepsilon^2V^2}}\\
		\text{As: }\sqrt{U^2+\varepsilon^2V^2}\rightarrow 0,\ \mu\pd{U}{y} \rightarrow \infty
	\end{gather*}
	
\end{document}